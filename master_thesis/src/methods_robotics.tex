%!TEX root = ../main.tex

\chapter{Methods}



\section{Source localisation}

\section{MLEM}

\section{Search strategy}
The maximum likelihood approach requires collecting as many measurements as possible to make the estimate more accurate.
At the same time, the drones should explore the unexplored area to increase chance that the estimation method won't miss any source of ionizing radiation.
The task for the mutirobotic system can be summarized as follows:
\begin{itemize}
  \item \textbf{exploration} - explore the least explored maps in the area of interest
  \item \textbf{exploitation} - acquire more measurements from areas, where the source of ionizing radiation is likely present
\end{itemize}
 
This general strategy is divided into subtasks: waypoint generation, task assignment and path planning.
\subsection{Waypoint generation}
s a first step, the $\lambda$ matrix (containing the current estimate of sources position) is processed by a local maximum filter. 
The maximum filter works by sliding a window of a specified size over the $\lambda$.
The central position of the sliding window is highlighted as local maxima if it is greater than all other values in the sliding window.

Each waypoint (local maxima) is assigned with a weight $w$.
The weight is defined as follows:
\begin{equation}
  w_{j} = \frac{\lambda_{j}}{s_{j_{normalised}}}
\end{equation},
where $s_{j_{normalised}} = \frac{s_{j}}{\max_{J}( s_{j})}$ is a sensitivity value for given position $s_{j}$ divided by.
Such formulation of $w_{j}$ prioritise the points with the highest current estimate of ionizing radiation, that are less explored (have lower sensitivity).

\subsection{Task assignment}

To fasten the search time, the waypoints are divided among all the \ac{UAV}s involved in the experiment.
This is done by KMeans clustering method with minimum number of samples within each cluster.

The KMeans algorithm is an iterative clustering algorithm, that can divide data points into $k$ clusters.
Each cluster is represented with a virtual point called centroid, $c_{k}, k = 1, ... , K$, where $K$ is the number of clusters.
The algorithm proceeds as follows:
/begin{itemize}
/end{itemize}




\subsection{Path planning}





collect more measurements
explore the least explored area

\subsection{Multirobotic approach}
there are two approaches - centralised and decentralised.
I use centralised.

\subsection{Task assignment}

\subsection{Clustering}
\subsection{Path planning}











