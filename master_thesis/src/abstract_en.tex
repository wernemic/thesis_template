%!TEX root = ../main.tex

\begin{changemargin}{0.8cm}{0.8cm}

~\vfill{}

\section*{Abstract}
\vskip 0.5em

This thesis presents a method for localizing multiple sources of ionizing radiation using a group of \acp{UAV}. 
These \acp{UAV} are equipped with miniature single-detector Compton camera radiation sensors, enabling them to estimate the directions towards high-energy gamma radiation sources.
  The proposed radiation mapping method (utilizing a \ac{MLE} principle) fuses the measurements from the Compton camera sensors to accurately estimate the positions of radioactive sources during the flight.
The properties of the used detector are approximated using Monte Carlo simulation techniques.
The estimation method is combined with an active search strategy that coordinates future action of the drones in order to improve the quality of estimate of sources position and minimize search time.
%In addition, an active search strategy is incorporated into the system, coordinating the future actions of the drones. 
%This strategy aims to improve the quality of the estimated source positions and minimize the search time by efficiently exploring the designated area of interest.
The proposed solution is evaluated on recorded real world data and in realistic simulator.





  \mycomment{
%The study of autonomous \acp{UAV} has become a prominent sub-field of mobile robotics.

This thesis presents a method for localization of multiple sources of ionizing radiation using a group of autonomous \acp{UAV} 
that are equipped with a miniature single-detector Compton camera radiation sensor capable of estimating directions towards a source of high-energy gamma radiation.
The proposed radiation mapping method (based on a \ac{MLE} principle) fuses the measurements from the Compton camera sensors and estimates the positions of radioactive sources during the flight.
Properties of the detector are approximated using Monte Carlo simulation.
The estimation method is combined with an active search strategy that coordinates future action of the drones in order to improve the quality of estimate of sources position and minimize search time.
The solution is evaluated on recorded real world data and in realistic simulator.
%The proposed estimation method based on maximum likelihood principle fuses the Compton camera measurements and estimates the positions of radioactive sources during the flight.
%The MiniPIX TPX3 Compton camera is can electronvolt
%The proposed solution utilizes a data fusion method based on \ac{MLEM} algorithm, which combines measurements from multiple \acp{UAV} and estimates the positions of radioactive sources during the flight.
%The fusion method 

%  for fusing the measurements
%and estimating the radiation source position during the flight.
}
\vskip 1em

{\bf Keywords} \Keywords

\vskip 2.5cm

\end{changemargin}
