%!TEX root = ../main.tex
\chapter{Conclusion\label{chap:conclusion}}
The goal of this thesis was to research, design, and implement an algorithm and software method for collaborative sensor fusion of
measured ionizing radiation data from a group of \ac{UAV}s.
The presented solution based on the \ac{MLEM} algorithm is able to localize multiple sources of ionizing radiation based on data acquired by a group of drones equipped with a miniature Compton camera. All the subtasks in the thesis assignment were fulfilled:
\begin{itemize}
  \item The author of the thesis familiarized himself with MRS UAV System and the principles of the Compton camera detector. 
The use of MRS UAV System was demonstrated during simulated and real-world experiments described in chapter \ref{chap:results}. 
An overview of principles of the Compton camera is presented in chapter \ref{chap:preliminaries}.
  \item A method for the localization of multiple sources of ionizing radiation using the Compton camera measurements was implemented. 
The online estimation method is based on the \ac{MLEM} algorithm, which was adapted to the proposed application.
Directional sensitivity of the \ac{pix} sensor was studied using Monte Carlo simulation.
    The proposed method takes into account the sensitivity of detection (how likely a potential source at certain position have been detected by the \ac{UAV}s during the flight), which improves the quality of estimate in scenarios with multiple sources of radiation with different emission activity.
The proposed localization method is presented in chapter \ref{chap:methods_estimation}.
  \item The proposed active search strategy for a small group of \ac{UAV} is presented in chapter \ref{chap:methods_robotics}.
    The centralized search strategy takes the current estimate of radioactive sources as an input and generates waypoints for the \ac{UAV}s in order to acquire more measurements and explored less explored parts of the area.         
    The generated waypoints are assigned to the individual drones and a non-colliding path connecting the waypoints is computed for each drone.
  \item The proposed estimation method and the search strategy was evaluated on recorded real-world data and in simulation, as presented in chapter \ref{chap:results}.
    %Experiments with recorded real data showed that the proposed method is abl
    The functionality of the whole system was demonstrated during outdoor experiments with real hardware and simulated sources of radiation 
    (in addition to the scope of the thesis assignment). 
\end{itemize}


\section{Future work}
The experiments with recorded real-world data showed that the real measurements acquired by the \ac{pix} sensor contain many outliers and noise in the measurements that significantly affected the quality of \ac{MLEM} reconstruction.
Experiments in simulation (with simulated noise) showed that the proposed active search strategy is able to overcome the noise in measurements by controlling the drones to get more data.
However, the combination of the proposed search strategy and the \ac{MLEM} estimation method could not be tested with real sources of ionizing radiation (due to organization reasons).
Future testing with real Caesium-137 sources is therefore needed for accurate evaluation.

The proposed localization method was developed for 2D radiation mapping in an outdoor open area.
A possible future extension is to fuse the Compton measurements with a 3D map of the environment (computed by some \ac{SLAM} method based on \ac{LiDAR} data).
The set of obstacles generated by the \ac{SLAM} method can be used as a set of possible sources locations in the \ac{MLEM} algorithm.
This would open a possibility to monitor radiation even in areas with obstacles or in the indoor environment.





%The stochastic nature of radioactive decay and limited sensing range of detectors (caused by properties of radiation spread in the environment) implies strong dependency between the drones' trajectories and quality of measurements.

%Testingi
%The proposed active search strategy helps to improve the quality of \ac{MLEM} estimate by controlling the drones based on the current estimate.
%However, the combination of the proposed radiation mapping method and search strategy was not tested on real radiation sources.



%The experiments showed that the quality of sources localization strongly depends on the measured data (given the stochastic nature of radioactive sources and limited sensing range of detectors caused by properties of ionizing radiation).

%therefore an active search strategy significantly 


\mycomment{
In this thesis, a method for collaborative sensor fusion of measured ionizing radiation data from a group of unmanned aerial vehicle was presented.


\alertwarning{Under construction}
The solution for the localization of multiple sources of ionizing radiation was presented.
The sensory fusion algorithm is based on the maximum likelihood expectation maximization approach.
To summarize, a small group of \ac{UAV}s equipped with Compton camera sensors are capable of detecting strong sources of ionizing radiation.

\section{Future work}
There are multiple possible improvements.
First of all, a more accurate model for the sensitivity of the Minipix3 sensor (with respect to the direction of incoming gamma particles) should be introduced.
One possible way might be a Monte Carlo simulation.
It should improve the precision of localization.
Secondly, the localization method might be extended into 3D.
An octomap obtained by an onboard lidar sensor might serve as a model of the terrain surface and possible source locations.
Thirdly, some kind of dosimetric data from the sensor might be used to estimate the "strength" of the source and shorten the time needed for localization since the number of particles causing the Compton effect in the Compton camera is relatively small.
Lastly, an active search method should be introduced to shorten the time of localization and control the group of \ac{UAV}s in an optimal way.

to say
it worked
method for mapping of multiple sources was introduced



possible extensions
3D map made with lidar data
%more accurate simulator

%needs testing in real life


TODO
%further testing
%validation in reality
}
