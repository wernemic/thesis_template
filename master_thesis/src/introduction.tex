%!TEX root = ../main.tex

\chapter{Introduction\label{chap:introduction}}

First, introduce the reader to the research topic.
Start with the most general view and slowly converge to the particular field, sub-field, and the challenges you face.
You can cite others' work here \cite{baca2021mrs}.

\section{Related works}

This section should contain related state-of-the-art works and their relation to the author's work.
We usually cite the original works like this \cite{benallegue2008high}.
You can also cite multiple papers at once like this \cite{baca2016embedded, baca2021mrs}.




\subsection{ MRS paper}
This thesis builds on the work of the MRS group (Faculty of Electrical engineering, CTU in Prague) in the field of radiation mapping. 
This paper \cite{baca2021gamma} presents a multi-robotic approach to a autonomous localisation of a compact source of gamma radiation. 
All unmanned aerial vehicles are equipped with compact MiniPIX TPX3 CdTe event camera, which is capable of measuring gamma particles and reconstructing Compton events. 
Output of the sensor are Compton cones are used for localisation in the following way:
in the first phase of autonomous exploration, the \ac{UAV}s are exploring the area in order to measure first compton cones. 
After first 8 cones are reconstructed, their intersection is computed using optimization methods (quadratic programming). 
This initial estimate is then incrementally updated using new measurements. 
The current estimate is always orthogonally projected to the newly measured cone. 
These measurements are fused using \ac{LKF}. 
The \ac{UAV}s are controlled in a way that they encircle the current estimate in order to measure more cones and update the \ac{LKF} estimate.

Using this approach, the group of drones is capable to localize single compact source of ionizing radiation. 
The source can be static of dynamic. 
However, this iterative method is not capable to localize multiple sources of radiation.
Once the drones detect one source of radioactive particles, they start encircling the current estimate and cannot find other sources the area.

The main purpose of this thesis is to improve the presented solution: introduce new method that would be capable to localize multiple sources of ionizing radiation and control group of \ac{UAV}s in order to  explore the whole area and maximize information gain given the specific sensor.

\section{TODO}
\begin{itemize}
\item clanek MRS ze ktereho vychazim
\item diplomka Petra Štibingera
\item ostatni robotické články týkající se robotickeho pruzkumu, leteckeho nebo pozemniho
\end{itemize}
\section{Contributions}

This section should describe the author's contributions to the field of research.

\section{Mathematical notation}

It is a good practice to define basic mathematical notation in the introduction.
See \reftab{tab:mathematical_notation} for an example.

\begin{table*}[!h]
  \scriptsize
  \centering
  \noindent\rule{\textwidth}{0.5pt}
  \begin{tabular}{lll}
    $\mathbf{x}$, $\bm{\alpha}$ & vector, pseudo-vector, or tuple\\
    $\mathbf{\hat{x}}$, $\bm{\hat{\omega}}$& unit vector or unit pseudo-vector\\
    $\mathbf{\hat{e}}_1, \mathbf{\hat{e}}_2, \mathbf{\hat{e}}_3$ & elements of the \emph{standard basis} \\
    $\mathbf{X}, \bm{\Omega}$ & matrix \\
    $\mathbf{I}$ & identity matrix \\
    $x = \mathbf{a}^\intercal\mathbf{b}$ & inner product of $\mathbf{a}$, $\mathbf{b}$ $\in \mathbb{R}^3$\\
    $\mathbf{x} = \mathbf{a}\times\mathbf{b}$ & cross product of $\mathbf{a}$, $\mathbf{b}$ $\in \mathbb{R}^3$\\
    $\mathbf{x} = \mathbf{a}\circ\mathbf{b}$ & element-wise product of $\mathbf{a}$, $\mathbf{b}$ $\in \mathbb{R}^3$ \\
    $\mathbf{x}_{(n)}$ = $\mathbf{x}^\intercal\mathbf{\hat{e}}_n$ & $\mathrm{n}^{\mathrm{th}}$ vector element (row), $\mathbf{x}, \mathbf{e} \in \mathbb{R}^3$\\
    $\mathbf{X}_{(a,b)}$ & matrix element, (row, column)\\
    $x_{d}$ & $x_d$ is \emph{desired}, a reference\\
    $\dot{x}, \ddot{x}, \dot{\ddot{x}}$, $\ddot{\ddot{x}}$ & ${1^{\mathrm{st}}}$, ${2^{\mathrm{nd}}}$, ${3^{\mathrm{rd}}}$, and ${4^{\mathrm{th}}}$ time derivative of $x$\\
    $x_{[n]}$ & $x$ at the sample $n$ \\
    $\mathbf{A}, \mathbf{B}, \mathbf{x}$ & LTI system matrix, input matrix and input vector\\
    \emph{SO(3)} & 3D special orthogonal group of rotations\\
    \emph{SE(3)} & \emph{SO(3)}~$\times~\mathbb{R}^3$, special Euclidean group\\
  \end{tabular}
  \noindent\rule{\textwidth}{0.5pt}
  \caption{Mathematical notation, nomenclature and notable symbols.}
  \label{tab:mathematical_notation}
\end{table*}
