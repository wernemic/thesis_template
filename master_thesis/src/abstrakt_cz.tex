%!TEX root = ../main.tex

\begin{changemargin}{0.8cm}{0.8cm}

~\vfill{}

\section*{Abstrakt}
\vskip 0.5em

\sloppy

Tato práce představuje metodu lokalizace více zdrojů ionizující radiace pomocí bezpilotních prostředků (dronů).
Tyto drony jsou vybaveny miniaturním detektorem fungujícím na pripcipu jednovrstvé Comptonovy kamery, který umožňuje odhadovat směry k zdrojům vysokoenergetického gama záření.
Navržená metoda pro mapování radiace (založená na principu maximální věrohodnosti) kombinuje měření z Comptonových kamer a za letu odhaduje pozice radioaktivních zdrojů.
Vlastnosti použitého senzoru jsou odhadnuty pomocí Monte Carlo simulace.
Metoda odhadu pozice zdrojů je je kombinována s aktivní prohledávací strategií, která koordinuje pohyb bezpilotních helikoptér za účelem zlepšení kvality odhadu pozice zdrojů radiace a minimalizace času potřebnému k jejich nalezení. 
Navžené řešení je otestováno na dříve naměřených datech ze skutečných senzorů a pomocí realistického simulátoru.

\vskip 1em

{\bf Klíčová slova} \KlicovaSlova

\vskip 2.5cm

\end{changemargin}
